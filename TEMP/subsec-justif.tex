\subsection{Justificativa da pesquisa} \label{subsec:justif}
    
    Geralmente, os sistemas produtivos possuem uma estrutura hierárquica bem definida para o planejamento e controle da produção. Essa hierarquia é dividida basicamente em três níveis de planejamento: longo prazo (estratégico), médio prazo (tático) e curto prazo (controle), conforme mostrado na Figura~\ref{fig:hierarquia}.
    
    \input{Introducao/fig:hierarquia}
    %% LONG: A função básica das ferramentas de planejamento estratégico de longo prazo mostradas na Figura 13.9 é estabelecer um ambiente de produção capaz de atender aos objetivos gerais da fábrica. PREVISÃO, PLANEJAMENTO DE CAPACIDADE E INTALAÇÕES, PLANEJAMENTO DE MÃO-DE-OBRA.
    
    \citeonline{Hopp2001} afirmam que, no planejamento estratégico a longo prazo (planejamento da produção), a função básica de suas ferramentas é de estabelecer um ambiente de produção favorável ao atendimento aos objetivos gerais da fábrica, onde as ferramentas são basicamente: a previsão, o planejamento de capacidade e instalações, e o planejamento de mão-de-obra. 
    
    %% INTERMEDIATE: As ferramentas táticas intermediárias na Figura 13.9 tomam os planos de longo prazo do nível estratégico, juntamente com informações sobre pedidos de clientes, para gerar um plano geral de ação que ajudará a fábrica a preparar a próxima produção (adquirindo materiais, alinhando subcontratantes, etc). PLANO MESTRE DE PRODUÇÃO (MPS), SEQUENCIAMENTO E PROGRAMAÇÃO.
    
    No nível tático (programação da produção) as ferramentas tomam os planos e informações gerados no nível estratégico para elaborar um plano geral de ação a fim de auxiliar na preparação da fábrica para a próxima produção. Dentre as ferramentas tem-se o Plano Mestre de Produção (MPS), e Sequenciamento e Programação.
    
    %% SHORT: As ferramentas de baixo nível na Figura 13.9 controlam diretamente a planta. CONTROLE DO CHÃO DE FÁBRICA, RASTREAMENTO DA PRODUÇÃO, SIMULAÇÃO EM TEMPO REAL
    
    Por fim, no planejamento a curto prazo (execução), as ferramentas controlam diretamente a fábrica, ou seja, o controle da produção em si. Módulos de controle de chão de fábrica, Rastreamento da produção, e Simulações em tempo real são exemplos de ferramentas utilizadas nesse nível mais baixo da hierarquia. Além disso, é neste nível que costumam ocorrer os \textit{feedbacks} para os níveis superiores de produção, como um ajuste no planejamento de capacidade no nível estratégico, ou no sequenciamento da produção no nível tático.
    
    %% PROBLEMAS DO FEEDBACK DEMORADO;
    %% COMPARAR A FIGURA 4A COM 4B, E FALAR QUE A IDEIA É GERAR O FEEDBACK MAIS PRÓXIMO;
    %\lipsum[1]
    
    Entretanto, o \textit{feedback} geralmente ocorre apenas quando o processo já está em operação no nível de execução, tendo pouca ou nenhuma interação entre os primeiros níveis de produção. Isso acaba gerando problemas no planejamento como um todo, que se agravam pois é uma prática comum ter departamentos ou equipes diferentes para cada fase da operação que não conversam entre si \cite{Hopp2001}.
    
    Como forma de evitar essa demora de troca de informações entre as operações, propõe-se a integração nos primeiros níveis de produção na hierarquia: o planejamento e a programação. Assim, a estrutura apresentada na Figura~\ref{subfig:usual} passe a ser executada conforme a Figura~\ref{subfig:prop}.
    
    \input{Introducao/fig:plan_prog}
    
    %% A INTEGRAÇÃO DO PLANEJAMENTO E PROGRAMAÇÃO PROPORCIONA QUE A ESTIMATIVA DE CAPACIDADE DADA PELA CLEARING FUNCTION NO PLANEJAMENTO SEJA UTILIZADA PELO LOT STREAMING NA PROGRAMAÇÃO, E QUE A PROGRAMAÇÃO REPORTA OS RESULTADOS PARA O PLANEJAMENTO;
    %% INTEGRANDO AS DUAS FUNÇÕES CONSEGUE-SE FAZER UM CÍRCULO VIRTUOSO DO PLANEJAMENTO COM A PROGRAMAÇÃO;
    %\lipsum[2-3]
    
    Isso significa que, ao invés de esperar o retorno da fase de produção para os ajustes serem realizados, as fases de planejamento e de programação da produção, por meio do uso das técnicas de \textit{clearing function} e \textit{lot streaming}, respectivamente, irão trabalhar as técnicas em conjunto. A integração do planejamento e programação proporciona que a estimativa de capacidade dada pela \textit{clearing function} no planejamento seja utilizado pelo \textit{lot streaming} na programação, e então esta retorna os resultados para o planejamento, conforme Figura~\ref{subfig:prop}.
    
    Com isso, integrando as duas funções consegue-se fazer um círculo virtuoso do planejamento e programação, a fim de manter a precisão e confiabilidade das estimativas de capacidade do sistema produtivo.
    
    %% O PLANEJAMENTO E PROGRAMAÇÃO DA PRODUÇÃO NORMALMENTE SÃO FEITOS EM ETAPAS SEPARADAS, O QUE PODE OCASIONAR PERDA DE INFORMAÇÃO E ATUALIZAÇÃO DO SISTEMA PRODUTIVO;
    %% 15.3 LINKING PLANNING AND SCHEDULING - FACTORY PHYSICS
    %% 13.5 - Factory
    %% CHAP 10 - JACOBS 2011 - ONE MANAGERIAL PROBLEM IS TO MATCH THE CAPACITY WITH THE PLANS: EITHER TO PROVIDE SUFFICIENT CAPACITY TO EXECUTE PLANS OR TO ADJUST PLANS TO MATCH CAPACITY CONSTRAINTS.
    %% A INTEGRAÇÃO DO PLANEJAMENTO E PROGRAMAÇÃO PROPORCIONA QUE A ESTIMATIVA DE CAPACIDADE DADA PELA CLEARING FUNCTION NO PLANEJAMENTO SEJA UTILIZADA PELO LOT STREAMING NA PROGRAMAÇÃO, E QUE A PROGRAMAÇÃO REPORTA OS RESULTADOS PARA O PLANEJAMENTO;
    %% INTEGRANDO AS DUAS FUNÇÕES CONSEGUE-SE FAZER UM CÍRCULO VIRTUOSO DO PLANEJAMENTO COM A PROGRAMAÇÃO;
    %% PLOTAR O DESENHO DO FLUXOGRAMA
    