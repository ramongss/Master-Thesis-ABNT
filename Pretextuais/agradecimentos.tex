\begin{center}
    \textbf{AGRADECIMENTOS}
\end{center}

Primeiramente à Deus por tudo que Ele proporcionou, de ter aberto os caminhos e me dado forças para vencer mais este desafio, sem Ele nada seria possível.

Aos meus familiares que, mesmo de longe, sempre me apoiaram e me incentivaram a seguir em frente de cabeça erguida e a buscar lugares mais altos. Em especial a minha mãe Ercilene Gomes dos Santos, minha vó Maria Assis dos Santos, e minha namorada Fernanda Costa Neves que em nenhum momento cessaram suas orações e deixaram de acreditar no meu sucesso.

Ao Prof. Raimundo J. B. de Sampaio, por ter me dado a oportunidade de estar trabalhando junto a ele, e ao longo do mestrado ter compartilhado de seu conhecimento e experiências, sempre buscando o meu crescimento profissional e pessoal, que com isso tornou-se possível a realização deste trabalho.

Ao Prof. Rafael R. G. Wollmann, pela disposição e paciência de estar me auxiliando nas minhas deficiências, e contribuindo com seus conhecimentos para o desenvolvimento da pesquisa.

Aos funcionários e demais professores da Pontifícia Universidade Católica do Paraná (PUCPR), em especial as secretárias Denise (PPGEPS) e Jane (PPGEM) pela paciência e solicitude em me atender e ajudar nas inúmeras vezes que precisei, e não medirem esforços para tal.

Aos meus amigos que ficaram em Belém torcendo, e aos novos que fiz durante esta caminhada que proporcionaram momentos agradáveis de diversão em meio a batalha. 

À CAPES pelo investimento da bolsa de estudos concedida permitindo que esta etapa da minha carreira profissional e acadêmica fosse concluída.