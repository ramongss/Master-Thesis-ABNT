\begin{abstract}
    \noindent Planejamento e programação da produção são etapas importantes do processo produtivo em um sistema de manufatura. Para garantir que o planejamento e programação de produção de um sistema produtivo sejam satisfatórios é necessário a estimação da capacidade desse sistema da maneira mais precisa possível. E para garantir essa precisão, são necessárias a utilização de regras de estimação tais como a \textit{Clearing Function}, que é uma função que estima capacidade quando o problema é modelado pela teoria de filas, e/ou a técnica de \textit{Lot Streaming}, que é uma técnica da programação que subdivide um lote de produção em sublotes menores para serem processados em sobreposição (\textit{overlapping}). O planejamento e a programação da produção são geralmente executados em etapas separadas, o que acaba gerando demora na troca de informações entre as operações desses níveis. Integrar essas etapas da produção significa fazer um círculo virtuoso entre o planejamento e a programação, uma vez que as estimativas de capacidade fornecidas pela \textit{Clearing Function}, no planejamento, são repassadas à programação que atualiza o \textit{Lot Streaming}, que por sua vez retorna o \textit{feedback} ao planejamento, buscando garantir a precisão e confiabilidade das estimativas. Este trabalho de pesquisa tratará do problema de integrar o Planejamento da Produção com a Programação da Produção por meio da compatibilização das estimativas de capacidade dadas pelas regras de \textit{Clearing Function} e pelas técnicas de \textit{Lot Streaming}. 
    \\ [5mm]
    \textbf{Palavras-chaves:} Planejamento da Produção, Programação da Produção, \textit{Lot Streaming}, \textit{Clearing Function}.
\end{abstract}