\begin{abstract}
    \noindent Planejamento e programação da produção são etapas importantes do processo produtivo em um sistema de manufatura. Para garantir que o planejamento e programação de produção de um sistema produtivo sejam satisfatórios é necessário a estimação da capacidade desse sistema da maneira mais precisa. Para garantir essa precisão, são necessárias a utilização de regras de estimação tal como o \textit{lot streaming}, que é uma técnica que subdivide um lote de produção em sublotes menores, para que operações consecutivas possam ser processadas em sobreposição (\textit{overlapping}), isto é, para que possam ser processadas em paralelo. Para problemas gerais, ou seja, sistemas de produção com $m$ máquinas e $n$ sublotes, o problema se torna extremamente complexo. Sistemas de produção desequilibrados geralmente são menos eficientes que sistemas equilibrados. Trabalhar com sistemas desequilibrados e complexos além de caros, necessitam de um conhecimento técnico específico e domínio das técnicas de \textit{lot streaming}, uma vez que não há uma regra geral para problemas com $m$ máquinas e $n$ sublotes. Este trabalho tratou de propor uma teoria empírica que estimasse a capacidade utilizando as técnicas de \textit{lot streaming}, e que fosse uma alternativa mais simples e barata para problemas mais gerais e complexos. Numerosos experimentos numéricos realizados na fase inicial desse estudo sugeriram que sistemas desequilibrados podiam ser representados usando-se sistemas equilibrados equivalentes em certo sentido específico, com um erro que que podia ser estimado a priori. Então planejamos e executamos um experimento numérico para testar a seguinte hipótese: O \textit{makespan} de sistemas de manufatura com $m$ máquinas e $n$ lotes com tempos distintos $p_i$ de processamento em cada máquina $i$ podem ser estimados adequadamente usando-se um tempo médio $p$ para cada máquina? Para responder a essa questão foi planejado e executado um planejamento estatístico do seguinte modo. Os experimentos foram divididos em 9 grupos. Com a utilização do \textit{Software R} foram gerados cenários para cada grupo. São os grupos, sistemas com: 5 máquinas e variação nos tempos de processamentos de até 20\%; 5 máquinas e variação de até 50\%; 5 máquinas e variação de até 70\%; 10 máquinas e variação de até 20\%; 10 máquinas e variação de até 50\%; 10 máquinas e variação de até 70\%; 15 máquinas e variação de até 20\%; 15 máquinas e variação de até 50\%; e, 15 máquinas e variação de até 70\%. Para cada grupo foram gerados 100 cenários com tempos de processamento aleatórios. Para cada um dos cenários aleatórios foi gerado um cenário equilibrado equivalente. Comparando os resultados dos tempos de completude de cada cenário, foi possível observar que sistemas mais equilibrados possuem uma diferença menor que 8\%. O trabalho concluiu que a alternativa oferece resultados mesmo não sendo ótimos, por um custo computacional extremamente baixo. E portanto pode-se construir uma teoria empírica de natureza prática que substitua a difícil tarefa de resolver problemas de \textit{makespan} com $m$ máquinas e $n$ lotes com tempos $p_i$ de processamento de cada máquina. Em conclusão o trabalho apresentado sugere uma teoria empírica para o problema geral de \textit{makespan} com $m$ máquinas e $n$ sublotes com tempos $p_i$ de processamento na máquina $i$ com margem de erro conhecida, usando a teoria disponível para duas e três máquinas, $n$ sublotes, e $p_i$ tempos de processamento de cada máquina $i$. Essa contribuição tem grande relevância prática porque não existe uma solução analítica geral para esse problema de grande interesse dos sistemas de manufatura.   \\ [5mm]
    \textbf{Palavras-chaves:} Capacidade, \textit{Lot Streaming}, Planejamento da Produção, Programação da Produção, Simulação.
\end{abstract}

