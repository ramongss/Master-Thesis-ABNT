\documentclass[12pt, a4paper, english, brazil]{article} %% TIPO DO ARQUIVO
\usepackage[english, brazil]{babel}     %% LÍNGUA
\usepackage[utf8]{inputenc}             %% FORMATO DO TEXTO
\usepackage[top=3cm, left=3cm, right=2cm, bottom=2cm]{geometry} %% MARGENS
\usepackage{setspace}                   %% PERMITE ESPAÇO SIMPLES
\usepackage{graphicx,amsmath,dsfont}    %% GRÁFICOS, EQUAÇÕES E FONTES
\setlength{\parindent}{1.25cm}          %% RECUO DE 1.25
\usepackage{indentfirst}                %% RECUO DA PRIMEIRA LINHA
\renewcommand{\baselinestretch}{1.3}    %% ESPAÇAMENTO 1.5
\pagestyle{myheadings}                  %% PAGINAÇÃO ABNT
\usepackage{enumitem}                   %% PACOTE DE NUMERAÇÃO
\usepackage{lipsum}                     %% LOREM IPSUM
\usepackage[table,xcdraw]{xcolor}       %% TABELA COLORIDA
\usepackage{multirow}                   %% LINHAS MESCLADAS
\usepackage{nomencl}                    %% NOMENCLATURA
\renewcommand{\nomname}{Lista de Abreviaturas e Siglas} %% RENOMEAR LISTA DE NOMENCLATURA
\makenomenclature                       %% FAZER A LISTA DE ABREVIAÇÕES
\usepackage{tikz}                       %% PLOTAR GRÁFICOS TIKZ
\usetikzlibrary{arrows,positioning}     %% ARCOS NO TIKZ
\tikzset{
    %Define standard arrow tip
    >=stealth',
    %Define style for boxes
    punkt/.style={
           rectangle,
           rounded corners,
           draw=black, very thick,
           text width=6.5em,
           minimum height=2em,
           text centered},
    % Define arrow style
    pil/.style={
           ->,
           thick,
           shorten <=2pt,
           shorten >=2pt,}
}
\usepackage{subfigure}                  %% ATIVAR SUBFIGURAS
\PassOptionsToPackage{subfigure}{tocloft} %% EVITAR ERROS DE SUBFIGURAS
\usepackage{fixmetodonotes}             %% EVITAR ERROS DE SUBFIGURAS
\usepackage{hyperref}                   %% HYPERLINK
\hypersetup{
    %bookmarks=true,         % show bookmarks bar?
    unicode=false,          % non-Latin characters in Acrobat’s bookmarks
    pdftoolbar=true,        % show Acrobat’s toolbar?
    pdfmenubar=true,        % show Acrobat’s menu?
    pdffitwindow=false,     % window fit to page when opened
    pdfstartview={FitH},    % fits the width of the page to the window
    %pdftitle={My title},    % title
    %pdfauthor={Author},     % author
    %pdfsubject={Subject},   % subject of the document
    %pdfcreator={Creator},   % creator of the document
    %pdfproducer={Producer}, % producer of the document
    %pdfkeywords={keyword1, key2, key3}, % list of keywords
    pdfnewwindow=true,      % links in new PDF window
    colorlinks=true,        % false: boxed links; true: colored links
    linkcolor=black,        % color of internal links (change box color with linkbordercolor)
    citecolor=black,        % color of links to bibliography
    filecolor=black,        % color of file links
    urlcolor=black          % color of external links
}
\usepackage{tocloft}                    %% PONTOS NO SUMÁRIO
\renewcommand{\cftsecleader}{\cftdotfill{\cftdotsep}} %% PONTOS NO SUMÁRIO 
\usepackage[alf, abnt-emphasize=bf]{abntex2cite} %% CITAÇÃO PADRÃO ABNT
\usepackage{pdfpages}                   %% IMPORTAR PÁGINAS PDF
\usepackage{titlesec}                   %% HABILITAR OS TÓPICOS QUATERNÁRIOS
\titleclass{\subsubsubsection}{straight}[\subsection]
\newcounter{subsubsubsection}[subsubsection]
\renewcommand\thesubsubsubsection{\thesubsubsection.\arabic{subsubsubsection}}
\renewcommand\theparagraph{\thesubsubsubsection.\arabic{paragraph}} %% optional; useful if paragraphs are to be numbered
\titleformat{\subsubsubsection}
  {\normalfont\normalsize\bfseries}{\thesubsubsubsection}{1em}{}
\titlespacing*{\subsubsubsection}
{0pt}{3.25ex plus 1ex minus .2ex}{1.5ex plus .2ex}
\makeatletter
\renewcommand\paragraph{\@startsection{paragraph}{5}{\z@}%
  {3.25ex \@plus1ex \@minus.2ex}%
  {-1em}%
  {\normalfont\normalsize\bfseries}}
\renewcommand\subparagraph{\@startsection{subparagraph}{6}{\parindent}%
  {3.25ex \@plus1ex \@minus .2ex}%
  {-1em}%
  {\normalfont\normalsize\bfseries}}
\def\toclevel@subsubsubsection{4}
\def\toclevel@paragraph{5}
\def\toclevel@paragraph{6}
\def\l@subsubsubsection{\@dottedtocline{4}{7em}{4em}}
\def\l@paragraph{\@dottedtocline{5}{10em}{5em}}
\def\l@subparagraph{\@dottedtocline{6}{14em}{6em}}
\makeatother
\setcounter{secnumdepth}{4}
\setcounter{tocdepth}{4}
