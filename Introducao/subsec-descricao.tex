\subsection{Descrição do problema} \label{subsec:descricao}
Recentemente, \citeonline{Bozek2017} fizeram uma revisão dos estudos com \textit{lot streaming}, apontando que as técnicas de \textit{lot streaming} são objeto de pesquisa há anos \cite{Potts1989, R.Baker1990475, Baker1993}, e por conta de sua importância prática e complexidade em modelagem e otimização, as técnicas ainda são utilizadas hoje em dia. As técnicas são utilizadas em conjunção com o \textit{flow shop} \cite{Potts1989, Liu2003, Biskup2006, Tseng2008, SancarEdis2009, Defersha2010, Defersha2012, Ventura2013, Mortezaei2014}, o \textit{job shop} \cite{Low2004773, Chan2008321, Petrovic2008275,SancarEdis2009442, Lei20134930} e o FJS (\textit{job shop} flexível) \cite{Bai2009, Defersha20122331, Calleja201493, Demir20143905, Rohaninejad2015, Bozek201621}. 

Os mesmos autores ainda afirmam que os trabalhos com formulações com sublotes iguais \cite{Potts1989, Chan2004472, Chan2008321, Low2004773, Tseng2008, SancarEdis2009442, Pan2012166, Ventura2013, Calleja201493} ou sublotes consistentes \cite{Potts1989, Low2004773, Chan2008321, SancarEdis2009, Bai2009, Defersha2012, Lei20134930, Demir20143905} são os mais comumente utilizados. Os trabalhos com problemas de \textit{lot streaming} mais complexos com sublotes variáveis são mais raros \cite{Liu2003, Biskup2006, Petrovic2008275, Defersha2010, Mortezaei2014}. 

Ainda, são analisados vários trabalhos com \textit{lot streaming} combinado com as mais diversas técnicas de dimensionamento de lotes (\textit{Lot Sizing}), e também com diversas abordagens de otimização. Em nenhum trabalho houve uma abordagem que estabelecesse uma regra geral para o \textit{lot streaming} ($m$ máquinas e $n$ sublotes), tamanha a complexidade do problema. A forma mais comum de se trabalhar com sistemas que possuem diversas máquinas é subdividir o problema de $m$ máquinas em $m-1$ problemas de 2 máquinas, uma vez que este está bem consolidado na literatura.

Além disso, problemas com sublotes variáveis, devido sua complexidade, são pouco explorados na literatura, apesar de que em casos reais, empresas de manufatura tendem a trabalhar com sublotes variáveis, por conta da configuração de seus maquinários. 

Diante disso, nota-se uma lacuna na literatura, onde não há uma abordagem as técnicas de \textit{lot streaming} que simplifique a sua aplicação em sistemas mais complexos, ou seja, com sublotes variáveis. A fim de preencher essa lacuna, este trabalho objetiva de propor uma alternativa por meio de uma teoria empírica que consiste em calcular um sistema balanceado no qual seus tempos de processamento sejam equivalentes aos do sistema desbalanceado em uso, de modo a apresentar uma solução satisfatória. Este sistema balanceado que é equivalente ao sistema desbalanceado será chamado a partir de agora de cenário equilibrado equivalente, e seus cálculos são apresentados nas Equações~\ref{eq:txtempo},~\ref{eq:txtempoeq} e~\ref{eq:peq}.

A questão norteadora da pesquisa portanto é: como resolver um problema complexo de \textit{lot streaming} de maneira simples e barata, que ao mesmo tempo apresente resultados satisfatórios de estimativa de capacidade e proporcione ao gestor identificar oportunidades de melhoria em seu sistema produtivo?

    