\subsection{Justificativa da pesquisa} \label{subsec:justif}

É conhecido na literatura que, geralmente, trabalhar com sistemas de produção equilibrados são mais fáceis de serem trabalhados em qualquer tipo de empresa, justamente por não necessitar de um grande conhecimento técnico para a realização do planejamento e da programação, por consequência disso se torna extremamente barato quando comparado aos sistemas desequilibrados. Além disso, sistemas desequilibrados, quando não planejados de maneira adequada, tendem a gerar sobrecarga em alguma estação de trabalho do sistema, isto é, gargalos. Portanto é necessário analisar o balanceamento da linha de produção para que o processo de produção ocorra em fluxo contínuo \cite{Syahputri2018}.

Balancear a capacidade de produção no chão-de-fábrica, isto é, equilibrar o sistema de produção, pode eventualmente eliminar ou reduzir os efeitos negativos dos gargalos na eficiência total da fábrica \cite{Varela2017}.

Entretanto, em casos reais, empresas de manufatura geralmente acabam trabalhando com sistemas desequilibrados em seus chãos-de-fábrica, o que acarreta em um desafio para o gestor tanto no planejamento de produção quanto principalmente na programação da produção.

Portanto, oferecer uma alternativa mais fácil e mais barata ao planejamento e programação da produção que diminua os efeitos do sistema desequilibrado é de extrema relevância tanto no ponto de vista prático do dia-a-dia da fábrica quanto para a academia científica. 

O trabalho apresenta uma teoria empírica onde é possível estimar um sistema equilibrado equivalente a partir de um sistema desequilibrado, de modo que pode-se comparar o quanto aquele sistema pode melhorar em relação a alternativa apresentada, cabendo ao gestor avaliar se a diferença de desempenho entre os sistemas é válida ou não à aplicação.