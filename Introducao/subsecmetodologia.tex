\subsection{Procedimentos metodológicos} \label{subsec:metod}
    \subsubsection{Classificação da pesquisa}
    Segundo \citeonline{Gil2017}, as pesquisas podem se referir aos mais diversos objetos e perseguir objetivos muito distintos. Classificar uma pesquisa se torna essencial para uma melhor organização dos fatos e, por consequência, seu entendimento.
    
    Portanto, esta pesquisa foi classificada utilizando alguns critérios como quanto à finalidade a pesquisa classifica-se como aplicada, uma vez que a pesquisa é voltada à aquisição de conhecimentos com vistas à aplicação numa situação específica; quanto à natureza dos dados classifica-se como pesquisa quantitativa por utilizar-se de métodos numéricos para a análise dos dados; e, quanto ao método a pesquisa é dedutiva, estatística e experimental.
    
    \subsubsection{Etapas da pesquisa}
    A pesquisa se deu seguindo as seguintes etapas:
    \begin{itemize}
        \item Levantamento do referencial teórico sobre o tema da pesquisa;
        \item Geração de cenários aleatórios de sistemas produtivos para simulações;
        \item Cálculo dos cenários equilibrados equivalentes aos cenários gerados;
        \item Simulações dos cenários de sistemas de produção utilizando as técnicas de \textit{lot streaming};
        \item Comparação dos resultados gerados pelos cenários aleatórios e seus equivalentes equilibrados;
        \item Avaliação e análise dos resultados gerados.
    \end{itemize}
    