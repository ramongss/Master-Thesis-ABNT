\section{Conclusões} \label{sec:conclusoes}

As técnicas de \textit{Lot Streaming} são uma excelente ferramenta tanto para programar a produção em lotes, quanto para estimar a capacidade produtiva daquele sistema. Contudo, a aplicação em sistemas de produção com $m$ máquinas é extremamente complexa, ainda mais se tratando de sistemas desbalanceados. Geralmente empresas de manufatura buscam trabalhar com sistemas mais próximos possíveis do equilíbrio. Portanto em um caso prático real, existe a necessidade do gestor possuir de um conhecimento técnico específico para programar a produção de maneira ótima.

Entretanto, como o trabalho apresentou, existe uma alternativa viável extremamente mais barata e mais fácil de ser aplicada. Foi observado que nos sistemas pouco desequilibrados a diferença para o ideal é de até 8\%, o que dependendo do caso pode ser uma diferença ínfima, onde trabalhando com o modelo mais simples não se obtém o ótimo, mas um resultado consideravelmente bom. 

O trabalho apresentado sugere uma teoria empírica para o problema geral de \textit{makespan} com $m$ máquinas e $n$ sublotes com tempos $p_i$ de processamento na máquina $i$ com margem de erro conhecida, usando a teoria disponível para duas e três máquinas, $n$ sublotes, e $p_i$ tempos de processamento de cada máquina $i$. Essa contribuição tem grande relevância prática porque não existe uma solução analítica geral para esse problema de grande interesse dos sistemas de manufatura.

A teoria empírica apresentada por este trabalhado faz com que o gestor ou responsável pela produção possa optar por operar o sistema produtivo como um sistema equilibrado e ainda sim obter resultados satisfatórios. É importante ressaltar novamente a facilidade e o custo computacional baixíssimo para operar desta forma, uma vez que não necessita de um conhecimento técnico específico ou dominar completamente as técnicas de \textit{Lot Streaming} para tal. 

A proposta apresentada pelo trabalhado é muito válida pois a maioria das empresas de manufatura buscam balanceamento na linha de produção para garantir o fluxo contínuo. O gestor ao se utilizar da teoria apresentada estima a capacidade do seu sistema de maneira satisfatória, e ainda pode ter uma perspectiva de melhoria na configuração de sua linha produtiva. 

O trabalho além de mostrar a importância e vantagens de se trabalhar com linhas de produção balanceadas por causa do seu desempenho, também mostra o quão ineficientes podem ser linhas de produção extremamente desbalanceadas. 

Ademais, o modelo apresentado no trabalho foi validado por conta das premissas cumpridas por ele, como a questão de sistemas equilibrados serem geralmente mais eficientes que sistemas desequilibrados, e de que no \textit{Lot Streaming} o tempo de completude do processo diminui conforme aumentasse o número de sublotes formando uma assintótica decrescente.


    \subsection{Limitações da pesquisa e propostas futuras}
    
    O autor deste trabalho não teve a oportunidade de realizar uma aplicação prática com dados reais. Apesar da pesquisa ter feito centenas de simulações de cenários aleatórios, seria interessante coletar dados reais de um sistema real em funcionamento, aplicar a teoria apresentada e comparar os resultados para validar se a teoria é realmente compatível com a realidade.
    
    Outro ponto foi que a pesquisa não considerou as técnicas de \textit{Lot Streaming} com os tempos de transporte entre máquinas. Uma proposta seria adicionar transportadores entre os pares de máquinas e comparar o quanto o desempenho é afetado com isso. 
    
    Por fim, uma terceira proposta seria a de comparar diferentes técnicas de programação e mensuração de tamanho de lotes (\textit{Lot Sizing}) com as técnicas de \textit{Lot Streaming}, a fim de avaliar qual técnica é mais adequada para determinada situação e/ou quanto o \textit{Lot Streaming} pode melhorar, etc.  